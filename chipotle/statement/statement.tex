\documentclass{oci}
\usepackage[utf8]{inputenc}
\usepackage{lipsum}

\title{La biblioteca perdida de la cultura Chipotle}

\begin{document}
\begin{problemDescription}
Tras una larga búsqueda, los científicos de la Organización de Culturas
Indígenas (OCI) han podido finalmente encontrar las ruinas de una antigua
biblioteca considerada hasta el momento perdida.
Entre las ruinas encontraron una basta colección de textos pertenecientes a la
cultura Chipotle, escritos en lo que pareciese ser Chipotle antiguo, un idioma
ancestral y aún desconocido.

Luego de estudiar los textos por algún tiempo, los científicos de la OCI notaron
que ninguno de estos contiene espacios y por lo tanto es imposible determinar
inmediatamente donde comienza y termina cada palabra.
Al parecer, la ancestral cultura Chipotle separaba las palabras a partir del
contexto.

Para la suerte de los científicos, entre los textos también se encontró un
diccionario con la lista de todas las palabras en el Chipotle antiguo.
Usando este diccionario, pensaron los científicos, sería sencillo separar los
textos en sus palabras constituyentes.

Al poco tiempo, los científicos se dieron cuenta que dado un texto y un
diccionario puede haber muchas formas de separar el texto en una secuencia de
palabras en el diccionario.
Por ejemplo, si el diccionario contiene las palabras \texttt{am, ame, ames,
  amet, eta, es, mes} y \texttt{tames}, entonces el texto \texttt{ametames}
puede separarse en palabras pertenecientes al diccionario de 4 formas distintas.

$$\{\texttt{am eta mes},\ \ \  \texttt{ame tames},\ \ \  \texttt{amet ames},\ \ \  \texttt{amet am es}  \}$$

% Otras separaciones como \texttt{am et ames} o \texttt{amet a mes} no son
% válidas, pues no 

Dependiendo de la cantidad de formas en que un texto puede separarse puede que
valga la pena intentarlo.
Es por esto que los científicos de la OCI están interesados en determinar de
cuantas formas posibles puede un texto ser separado en una secuencia de palabras
en el diccionario.
`? Podrías ayudarlos?
\end{problemDescription}

\begin{inputDescription}
  La primera línea de la entrada contiene un entero $D$ ($0 < D \leq 100$) que
  describe la cantidad de palabras en el diccionario.
  A continuación, cada una de las siguientes $D$ líneas contiene una cadena de
  caracteres correspondiente a una palabra en el diccionario.
  El diccionario $\textbf{no}$ contendrá palabras repetidas y cada una de las
  palabras tendrá largo al menos 1 y a lo más 10.
  Finalmente, la última línea de la entrada contiene una cadena de caracteres de
  largo $T$ ($0 < T \leq 10^5$) correspondiente al texto que quiere separase en
  palabras.

  Tanto las palabras en el diccionario como el texto estarán formados
  únicamente por caracteres en minúscula del alfabeto inglés (no contiene la
  ``ñ'').
\end{inputDescription}

\begin{outputDescription}
  La salida debe corresponder a un entero mayor o igual que 0 correspondiente a
  la cantidad de formas en que puede separarse el texto como una secuencia de
  palabras en el diccionario.
  Como este número puede ser muy grande se pide entregarlo módulo 1000009.
\end{outputDescription}

\begin{scoreDescription}
  \score{20} Se probarán varios casos donde todas las palabras en el diccionario
  tienen el mismo largo.
  \score{35} Se probarán varios casos donde el tamaño del texto es menor o igual
  que 100 ($T \leq 100$).
  \score{45} Se probarán varios casos sin restricciones adicionales.
\end{scoreDescription}

\begin{sampleDescription}
\sampleIO{sample-1}
\sampleIO{sample-2}
\sampleIO{sample-3}
\end{sampleDescription}

\end{document}
