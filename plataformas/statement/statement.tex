\documentclass{oci}
\usepackage[utf8]{inputenc}
\usepackage{lipsum}

\title{Plataformas}

\begin{document}
\begin{problemDescription}
Ociman debe escalar una montaña para conseguir una espada legendaria.

En la ladera de la montaña, existe un conjunto de plataformas horizontales sobre las que Ociman puede caminar y entre las cuales puede saltar, siempre que sea capaz de alcanzarlas.
% TODO dibujo que muestre un set de plataformas simples, que coincida con uno de los ejemplos de entrada/salida.
Ociman puede saltar de la plataforma $i$ a altura $h_i$ a la plataforma $j$ a altura $h_j$ si:
$$h_i - h_j \ge 25-(x-5)^2$$
donde $x$ es la distancia horizontal entre plataformas.
% TODO considerar la opción de entregarles una función que dado dx retorna el máximo dy.
% Vista de la función en https://www.wolframalpha.com/input/?i=25+-+(%7Cx%7C-5)%5E2
¿Puede llegar Ociman a la espada? De ser así, ¿cuántos saltos debe realizar Ociman para llegar a la espada legendaria?

\end{problemDescription}

\begin{inputDescription}
La primera línea de la entrada contiene dos enteros $N$ y $S$, que corresponden respectivamente al número de plataformas y la plataforma donde se encuentra la espada ($1 \le S \le N$).
Ociman siempre comienza en la plataforma número $1$.

Las siguientes $N$ líneas describen a las plataformas en orden.
Cada una de ellas contiene tres enteros $W$, $X$ e $Y$, que corresponden respectivamente al ancho de la plataforma, al extremo izquierdo de la plataforma y a la altura a la que se encuentra la plataforma.
\end{inputDescription}

\begin{outputDescription}
Si Ociman puede llegar a la espada, entonces imprima una línea con un entero: el mínimo número de saltos que debe realizar para lograrlo. Si Ocimar no puede llegar a ella, imprima la palabra \verb-Inalcanzable- en una línea.
\end{outputDescription}

\begin{scoreDescription}
  % TBD distribución de puntajes
  \score{10} $1 \le N \le 100$. Todas las plataformas tienen $W=Y=1$.
  % Considerar caso donde es posible saltar entre plataforma i e i+2 sin tocar i+1 y similares. Considerar caso donde el salto es hacia la izquierda.
  \score{10} $1 \le N \le 100$. Todas las plataformas tienen $W=1$.
  % Debe considerar plataformas en ambos sentidos.
  \score{10} $1 \le N \le 10$. Todas las plataformas tienen $1 \le W \le 10$.
  % Puede probar desde cada X en la plataforma de origen a cada Y en la plataforma de destino (O(N^2 * W^2)).
  \score{10} $1 \le N \le 1000$. Todas las plataformas tienen $1 \le W \le 1000$.
  % Debe encontrar la menor distancia entre dos plataformas (O(N^2)).
\end{scoreDescription}

\begin{sampleDescription}
\sampleIO{sample-1}
\sampleIO{sample-2}
\end{sampleDescription}

\end{document}
