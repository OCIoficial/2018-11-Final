\documentclass{oci}
\usepackage[utf8]{inputenc}
\usepackage{lipsum}

\title{Plataformas}

\begin{document}
\begin{problemDescription}
Ociman está a cargo del centro de operaciones de la OCI, donde se coordina la final de la competencia y se revisan los problemas.
Para llegar a este lugar sin ser visto, Ociman ha optado por usar un pogo stick, saltando de edificio en edificio por la ciudad (cuyo tamaño es de $N \times N$).

Para simplificar el problema, Ociman ha determinado que saltará de edificio en edificio, donde visto desde arriba, cada edificio es de tamaño $1 \times 1$, hasta llegar al edificio de la OCI.
Sin embargo, los saltos de Ociman están limitados por una distancia $d$: con cada salto puede moverse hasta $d$ hacia el norte, $d$ hacia el sur, $d$ hacia el oeste y $d$ hacia el este.

% TODO dibujo que muestre un set de plataformas simples, que coincida con uno de los ejemplos de entrada/salida.

¿Podrá Ociman llegar al edificio de la OCI o deberá buscar otro plan?
Si puede, cuál es el número mínimo de saltos que debe realizar para llegar a éste?

\end{problemDescription}

\begin{inputDescription}
La primera línea de la entrada dos enteros $N$ y $d$, que corresponden respectivamente a las dimensiones de la ciudad ($1 < N$) y la distancia máxima que puede saltar Ociman ($1 \le d \le N - 1$).

La segunda línea contiene dos enteros $U$ y $V$, que corresponden a las coordenadas de la posición inicial de Ociman ($1 \le U, V \le N$).

La tercera línea contiene dos enteros $X$ e $Y$, que corresponden a las coordenadas del edificio de la OCI ($1 \le X, Y \le N$).

Las siguientes $N$ líneas describen la ciudad.
Cada una de ellas contiene $N$ números $B_{i,j}$ separados por espacios, donde $B_{i,j}$ es el $j$-ésimo número de la $i$-ésima línea.
$B_{i,j}$ es 1 si hay un edificio donde Ociman puede saltar, y 0 si no.
Se garantiza que siempre tanto $B_{U,V}$ como $B_{X,Y}$ son 1.
\end{inputDescription}

\begin{outputDescription}
Si Ociman puede llegar al edificio de la OCI en pogo stick, imprima una línea con un entero: el mínimo número de saltos que debe realizar para lograrlo. Si Ociman no puede llegar a él, imprima la palabra \verb-Inalcanzable- en una línea.
\end{outputDescription}

\begin{scoreDescription}
  % TBD distribución de puntajes
  \score{10} $1 \le N \le ?$. $B_{i,j}$ es siempre 1.
  \score{10} $1 \le N \le ?$. No hay restricciones adicionales.
\end{scoreDescription}

\begin{sampleDescription}
\sampleIO{sample-1}
\sampleIO{sample-2}
\end{sampleDescription}

\end{document}
