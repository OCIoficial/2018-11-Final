\documentclass{oci}
\usepackage[utf8]{inputenc}
\usepackage{lipsum}

\title{Diccionario}

\begin{document}
\begin{problemDescription}
Investigadores de la Organización de Culturas Interesantes (OCI) descubren un antiguo texto escrito en un idioma ancestral. Con mucho esfuerzo logran identificar cada símbolo y caen en la cuenta de que el texto entero está escrito sin espacios.

Al parecer, la gente de esta cultura simplemente separaba sus frases por contexto. Por suerte, los investigadores encuentran otro texto que parece ser un diccionario. ¿puedes ayudarles a saber cuantas maneras hay de interpretar el texto como una secuencia de palabras del diccionario?

Por ejemplo, si el diccionario es \texttt{[amet, eta tames mes, ames, am, es]}, entonces el texto \texttt{ametames} puede interpretarse de cuatro formas: 

$$\{\texttt{am eta mes},\ \ \  \texttt{ame tames},\ \ \  \texttt{amet ames},\ \ \  \texttt{amet am es}  \}$$


\end{problemDescription}

\begin{inputDescription}
La entrada consiste de un número variable de líneas. la primera línea contiene un entero $N \geq 1$ que describe el número de palabras del diccionario. Las siguiente $N$ líneas contienen una cadena de caractéres que representa una palabra del diccionario. El diccionario $\textbf{no}$ contrendrá palabras repetidas. Finalemente, la última línea de la entrada contiene una cadena de catacteres que codifica el texto a decodificar. 

Cada cadena de caracteres está compuesta por las letras $\texttt{a},\texttt{b},\dots,\texttt{z}$ (ASCII 097 a 122). Además, cada una tiene al menos un caracter y máximo $10^5$.
\end{inputDescription}

\begin{outputDescription}
Una línea conteniendo un entero con la cantidad de formas de interpretar el texto como una secuencia de palabras del diccionario.
\end{outputDescription}

\begin{scoreDescription}
  \score{20} Todas las palabras del diccionario son del mismo tamaño. 
  \score{20} Las palabras del diccionario tienen largo máximo 10 y $N \leq 10$.
  \score{60} $N \leq 100$
\end{scoreDescription}

\begin{sampleDescription}
\sampleIO{sample-1}
\sampleIO{sample-2}
\end{sampleDescription}

\end{document}
